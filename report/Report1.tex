\documentclass[12pt,a4paper]{report}
\usepackage[utf8]{inputenc}
\usepackage[russian]{babel}
\usepackage[OT1]{fontenc}
\usepackage{amsmath}
\usepackage{amsfonts}
\usepackage{amssymb}
\usepackage{graphicx}
\usepackage{cmap}					% поиск в PDF
\usepackage{mathtext} 				% русские буквы в формулах
%\usepackage{tikz-uml}               % uml диаграммы

% TODOs
\usepackage[%
  colorinlistoftodos,
  shadow
]{todonotes}

% Генератор текста
\usepackage{blindtext}

%------------------------------------------------------------------------------

% Подсветка синтаксиса
\usepackage{color}
\usepackage{xcolor}
\usepackage{listings}
 
 % Цвета для кода
\definecolor{string}{HTML}{B40000} % цвет строк в коде
\definecolor{comment}{HTML}{008000} % цвет комментариев в коде
\definecolor{keyword}{HTML}{1A00FF} % цвет ключевых слов в коде
\definecolor{morecomment}{HTML}{8000FF} % цвет include и других элементов в коде
\definecolor{captiontext}{HTML}{FFFFFF} % цвет текста заголовка в коде
\definecolor{captionbk}{HTML}{999999} % цвет фона заголовка в коде
\definecolor{bk}{HTML}{FFFFFF} % цвет фона в коде
\definecolor{frame}{HTML}{999999} % цвет рамки в коде
\definecolor{brackets}{HTML}{B40000} % цвет скобок в коде
 
 % Настройки отображения кода
\lstset{
language=C, % Язык кода по умолчанию
morekeywords={*,...}, % если хотите добавить ключевые слова, то добавляйте
 % Цвета
keywordstyle=\color{keyword}\ttfamily\bfseries,
stringstyle=\color{string}\ttfamily,
commentstyle=\color{comment}\ttfamily\itshape,
morecomment=[l][\color{morecomment}]{\#}, 
 % Настройки отображения     
breaklines=true, % Перенос длинных строк
basicstyle=\ttfamily\footnotesize, % Шрифт для отображения кода
backgroundcolor=\color{bk}, % Цвет фона кода
%frame=lrb,xleftmargin=\fboxsep,xrightmargin=-\fboxsep, % Рамка, подогнанная к заголовку
frame=tblr
rulecolor=\color{frame}, % Цвет рамки
tabsize=3, % Размер табуляции в пробелах
 % Настройка отображения номеров строк. Если не нужно, то удалите весь блок
numbers=left, % Слева отображаются номера строк
stepnumber=1, % Каждую строку нумеровать
numbersep=5pt, % Отступ от кода 
numberstyle=\small\color{black}, % Стиль написания номеров строк
 % Для отображения русского языка
extendedchars=true,
literate={Ö}{{\"O}}1
  {Ä}{{\"A}}1
  {Ü}{{\"U}}1
  {ß}{{\ss}}1
  {ü}{{\"u}}1
  {ä}{{\"a}}1
  {ö}{{\"o}}1
  {~}{{\textasciitilde}}1
  {а}{{\selectfont\char224}}1
  {б}{{\selectfont\char225}}1
  {в}{{\selectfont\char226}}1
  {г}{{\selectfont\char227}}1
  {д}{{\selectfont\char228}}1
  {е}{{\selectfont\char229}}1
  {ё}{{\"e}}1
  {ж}{{\selectfont\char230}}1
  {з}{{\selectfont\char231}}1
  {и}{{\selectfont\char232}}1
  {й}{{\selectfont\char233}}1
  {к}{{\selectfont\char234}}1
  {л}{{\selectfont\char235}}1
  {м}{{\selectfont\char236}}1
  {н}{{\selectfont\char237}}1
  {о}{{\selectfont\char238}}1
  {п}{{\selectfont\char239}}1
  {р}{{\selectfont\char240}}1
  {с}{{\selectfont\char241}}1
  {т}{{\selectfont\char242}}1
  {у}{{\selectfont\char243}}1
  {ф}{{\selectfont\char244}}1
  {х}{{\selectfont\char245}}1
  {ц}{{\selectfont\char246}}1
  {ч}{{\selectfont\char247}}1
  {ш}{{\selectfont\char248}}1
  {щ}{{\selectfont\char249}}1
  {ъ}{{\selectfont\char250}}1
  {ы}{{\selectfont\char251}}1
  {ь}{{\selectfont\char252}}1
  {э}{{\selectfont\char253}}1
  {ю}{{\selectfont\char254}}1
  {я}{{\selectfont\char255}}1
  {А}{{\selectfont\char192}}1
  {Б}{{\selectfont\char193}}1
  {В}{{\selectfont\char194}}1
  {Г}{{\selectfont\char195}}1
  {Д}{{\selectfont\char196}}1
  {Е}{{\selectfont\char197}}1
  {Ё}{{\"E}}1
  {Ж}{{\selectfont\char198}}1
  {З}{{\selectfont\char199}}1
  {И}{{\selectfont\char200}}1
  {Й}{{\selectfont\char201}}1
  {К}{{\selectfont\char202}}1
  {Л}{{\selectfont\char203}}1
  {М}{{\selectfont\char204}}1
  {Н}{{\selectfont\char205}}1
  {О}{{\selectfont\char206}}1
  {П}{{\selectfont\char207}}1
  {Р}{{\selectfont\char208}}1
  {С}{{\selectfont\char209}}1
  {Т}{{\selectfont\char210}}1
  {У}{{\selectfont\char211}}1
  {Ф}{{\selectfont\char212}}1
  {Х}{{\selectfont\char213}}1
  {Ц}{{\selectfont\char214}}1
  {Ч}{{\selectfont\char215}}1
  {Ш}{{\selectfont\char216}}1
  {Щ}{{\selectfont\char217}}1
  {Ъ}{{\selectfont\char218}}1
  {Ы}{{\selectfont\char219}}1
  {Ь}{{\selectfont\char220}}1
  {Э}{{\selectfont\char221}}1
  {Ю}{{\selectfont\char222}}1
  {Я}{{\selectfont\char223}}1
  {і}{{\selectfont\char105}}1
  {ї}{{\selectfont\char168}}1
  {є}{{\selectfont\char185}}1
  {ґ}{{\selectfont\char160}}1
  {І}{{\selectfont\char73}}1
  {Ї}{{\selectfont\char136}}1
  {Є}{{\selectfont\char153}}1
  {Ґ}{{\selectfont\char128}}1
  {\{}{{{\color{brackets}\{}}}1 % Цвет скобок {
  {\}}{{{\color{brackets}\}}}}1 % Цвет скобок }
}
 
 % Для настройки заголовка кода
\usepackage{caption}
\DeclareCaptionFont{white}{\color{сaptiontext}}
\DeclareCaptionFormat{listing}{\parbox{\linewidth}{\colorbox{сaptionbk}{\parbox{\linewidth}{#1#2#3}}\vskip-4pt}}
\captionsetup[lstlisting]{format=listing,labelfont=white,textfont=white}
\renewcommand{\lstlistingname}{Код} % Переименование Listings в нужное именование структуры


%------------------------------------------------------------------------------

\author{Е.~А.~Никитин}
\title{Сети ЭВМ и телекоммуникации}
\begin{document}
\listoftodos
\maketitle
\chapter{Задание}
Разработать приложение-клиент и приложение сервер, обеспечивающие функции обмена файлами.
\section{Функциональные требования}
\todo[inline]{Начать можно с этого}
Серверное приложение должно реализовы-
вать следующие функции:
1) Прослушивание определенного порта
2) Обработка запросов на подключение по этому порту от клиентов
3) Поддержка одновременной работы нескольких клиентов через меха-
низм нитей
4) Приём файла от клиента
5) Передача по запросу клиента списка файлов текущего каталога
6) Приём запросов на передачу файла и передача файла клиенту
7) Навигация по системе каталогов
8) Обработка запроса на отключение клиента
9) Принудительное отключение клиента
Клиентское приложение должно реализовывать следующие функции:
1) Установление соединения с сервером
2) Получение от сервера списка файлов каталога
3) Операции навигации по системе каталогов
4) Передача файла серверу
5) Приём файла от сервера
6) Разрыв соединения
\section{Нефункциональные требования}
Т.к. для выполнения данного задания требовалось изучить библиотеки, преднозначенные для работы с файловой
системой как ОС Windiws, так ОС Linux, задача была упрощена - сервер написан для работы на Linux.
Клиентское прложение создано для ОС Windows.
\section{Накладываемые ограничения}
Максимальный размер буфера - 1024, поэтому рабочая директория сервера, а также пути, которые будет указывать пользователь должны укладываться в данную длину.
Ограничение по одновременно подключенным пользователям - 10.
В некоторых функциях в качестве разделителя используется знак "|", поэтому его использование в названиях и путях
нежелательно, т.к. может привести к неправильной работе приложения.
\chapter{Реализация для работы по протоколу TCP}
\section{Прикладной протокол}
\label{protocol_tcp}
В данной реализации для некоторых команд был создан собственный протокол,
отличный от реализации отправки в несколько посылок, однако реализация протокола труднее в описании.

Описание команд:
1) view - получить дерево каталогов всего рабочего пространства. Функция рекурсивно проходит все вложенные
папки рабочей директории. Посылки отправляются в формате: "отступ"|"символ папки или файла"|"название". После использования данной команды текущая директори меняется на корень рабочей папки.
2)view_dir - команда получения списка фалов и папок, но внутри текущей директории. Используется протокол команды view.
3)get_dir - команда получения текущей директории. Сообщение приходит в виде пути в файловой системе.
4)ch_dir - изменение ткущего каталога. Послыка формируется в виде пути относительно текущей папки.
5)mk_dir - создание папки. Папка будет создана в текущем каталоге. Послыка формируется в виде названия папки, ответ в виде успешного/неуспешного создания.
6)rm_dir - команда удаления папки из текущего каталога. 
7)help - получение списка и формата команд.
8)upload - функция загрузки файла на сервер. Работа в следующей последовательности: пользователь задает путь -> отправляется размер файла -> отправляется название (как будет называться на сервере), на сервере формируется путь -> запоминается время начала отправки -> отправялется файл -> высчитывается время передачи.
9)download - функция загрузки с сервера. Работа в следующей последовательности: потправляется название файла -> принимается подтверждение открытия файла на сервере -> указывается путь, куда сохранять(формируется путь) -> приимается размер файла -> запоминается время начала отправки -> отправялется файл -> высчитывается время передачи.

\section{Архитектура приложения}
Особенности архитектуры и ограничения (желательно с графической схемой) 

\section{Тестирование}
\subsection{Описание тестового стенда и методики тестирования}
\subsection{Тестовый план и результаты тестирования}
1) Тест1: ввод команд без подключения
2) Тест2: обработка неверных команд
3) Тест3: неправильный путь при загрузке файла на сервер
4) Тест4: получение дерева каталогов
5) Тест5: получения файла с сервера
6) Тест6: отправка файла на сервер

Результаты тестирования:
\begin{figure}[h!]
				\center{\includegraphics[scale=0.8]{test_2cli}}
				\caption{Ввод команд без подключения}
				\label{img:test1}
			\end{figure}

\chapter{Реализация для работы по протоколу UDP}
\section{Прикладной протокол}
Протоколы команд совпадют с протоколами в реализации TCP, кроме команд upload и download, в которых добавлена проверка на целостность полученного файла. После передачи дополнительная проверка и отправление результата в виде строки "error"/"ок".

\section{Архитектура приложения}
Архитектура приложения относительно реализации TCP изменилась. Обработчик команд был обособлен в функцию диспетчера. До диспетчера устанавливается соединение и создание новых сокетов для клиентов.
Архитектура представлена на рисунке:

\section{Тестирование}
\subsection{Описание тестового стенда и методики тестирования}
\subsection{Тестовый план и результаты тестирования}
Тестирование проведено согласно плану тестирования TCP приложений. Результаты совпали.


\chapter{Выводы}
В результате работы были созданы клиентское и серверное приложения для собственного протокола взаимодествия. Было создано две реализации приложения для протоколов TCP и
UDP. Клиентское и серверное приложения были реализованы для двух
разных платформ: ОС Windows и Linux. При реализации использова-
лись стандартные сокеты. Реализации сокетов для использованных ОС
идентичны, портирование программ с одной платформы на другую вы-
полняется достаточно просто.
В результате работы была создана клиент-серверная система загрузки и приема файлов с сервера.
Файлы храянятся в общей папке, клиенты могут создавать, удалять папки, пермещаться по каталогу.
\section{Реализация для TCP}
Протокол TCP удобен для реализации пользовательских приложений,
так как обеспечивает установление соединения и надежную доставку па-
кетов. Протокол обечпечивает стабильное надежное соединение, поэто-
му при реализации своего протокола не требуется волноваться об этом.
Однако, эти дополнительные средства синхронизации требуют больше
времени на доставку, т.е. скорость передачи данных ниже чем в UDP.

\section{Реализация для UDP}
Протокол UDP удобен для реализации приложений, не требующих точ-
ной доставки пакетов. Он позволяет передавать данные с большей скоро-
стью, однако вероятность потери пакета при этом выше, чем в TCP. По-
этому, использовать данный протокол для реализации поставленной за-
дачи не очень удобно. Требуется использовать дополнительные инстру-
менты для подтверждения корректной доставки, т.е. каким-то образом
«симулировать» TCP. Это неудобно и неэффективно.
\chapter*{Приложения}
\section*{Описание среды разработки}
Серерное приложение реализовано на ОС Linux debian 3.16.0 (среда разработки QT Creator 3.5.0), клиентское на ОС Windows 7 (среда разработки Visual Stidio 2010). Срервер запускался на виртуальной машине, соединение через сетвеой мост.

\section*{Листинги}
\subsection*{Основной файл программы main.c}
%\lstinputlisting[]
%{/home/user/workspace/tcp_server/main.c}
%\subsection*{Файл сборки Makefile}
%\lstinputlisting[language=make,label={Makefile}]
%{/home/user/workspace/tcp_server/Makefile}
\todo[inline]{Не забыть вставить все исходники}
\end{document}